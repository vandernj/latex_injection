\documentclass[reqno]{article} %equation numbers on the right
\usepackage[utf8]{inputenc}
\usepackage{amsmath,amssymb,amsthm}
\usepackage[retainorgcmds]{IEEEtrantools}% better equation multilines
\usepackage{array}
%\usepackage{fullpage} //for when you want to not use just half the page
%\usepackage{listings}
%\usepackage{algpseudocode}
%\usepackage{graphicx}
%shorthand has no numbering
\theoremstyle{definition} \newtheorem*{defn}{Definition}
\theoremstyle{plain} \newtheorem*{thm}{Theorem}
\theoremstyle{plain} \newtheorem*{lem}{Lemma}
\theoremstyle{remark} \newtheorem*{rmk}{Remark}
%longhand does
\theoremstyle{definition} \newtheorem{definition}{Definition}[section]
\theoremstyle{plain} \newtheorem{theorem}{Theorem}[section]
\theoremstyle{plain} \newtheorem{lemma}{Lemma}[section]
\theoremstyle{remark} \newtheorem{remark}{Remark}[section]

%\DeclareMathOperator{\swag}{swag}
%\DeclareMathOperator*{\sweg}{sweg} //plays nicely with limits
\DeclareMathOperator{\e}{\varepsilon}
\DeclareMathOperator{\ZZ}{\mathbb{Z}} %sage style set names
\DeclareMathOperator{\QQ}{\mathbb{Q}}
\DeclareMathOperator{\NN}{\mathbb{N}}
\DeclareMathOperator{\RR}{\mathbb{R}}
\DeclareMathOperator{\CC}{\mathbb{C}}
\DeclareMathOperator{\FF}{\mathbb{F}}
\DeclareMathOperator{\st}{\textrm{ s.t. }} 
\def\phi{\varphi} %dont do this unless you're super lazy 

%keep this line at bottom
\usepackage[pdftex]{hyperref} %creates links to referenced eqns. 
\begin{document}
\title{Factoring Polynomials mod $p$}
\author{Ting Gong, Nick VanderLaan, and Nikhil Shankar}
\maketitle

\begin{abstract}
Let $f(x) = x^n + c_{n-1}x^{n-1} + \ldots c_1x + c_0$  be a monic polynomial function in $\mathbb{Z}/p\mathbb{Z}[x]$.  We say such a function can be factored if it can be written as: $f(x) = f_1(x) \cdot f_2(x)\cdot \ldots \cdot f_k(x)$  where each $f_i$ is in  $\mathbb{Z}/p\mathbb{Z}[x]$. We will explore when such factorizations are possible.
\end{abstract}

\section{Introduction}
By the Fundamental Theorem of Algebra, every non-constant, complex polynomial has a root in $\CC$, and therefore can be written as the product of linear (e.g. $f(x) = ax + b$) and constant terms. Over $\RR$, not every polynomial has a root, the prototypical example being $x^2 + 1$. However, it is known that we may write any real polynomial as the
product of linear and quadratic factors, so every polynomial of 
degree $\geq 3$ is reducible. Over $\QQ$ this is no longer the case, as $x^4 + 1$
admits neither linear nor quadratic factors. This paper concerns factoring over finite fields, focusing on the classification of irreducible monic polynomials. 

\subsection{Notation and Definitions}
\begin{enumerate}
\item We denote $\mathbb{F}_p$ as the finite field $\mathbb{Z}/p\mathbb{Z}$ where $p$ is a prime number.
\item $\mathbb{F}_p[x]$ is the polynomial ring over field $\mathbb{F}_p$.
\item An n-degree polynomial in $\mathbb{F}_p[x]$ can be written as $$f(x) = x^n + c_{n-1}x^{n-1}+ \ldots + c_1x+ c_0$$ where $c_0,c_1,\ldots, c_{n-1}\in \mathbb{F}_p$.
\end{enumerate}

\begin{definition}
A polynomial $f \in \FF_p [x]$ is said to be \textbf{reducible} if we can write $f = gh$ where $g, h \in \FF_p [x]$ non-constant polynomials. If no such factorization exists, 
$f$ is said to be \textbf{irreducible}. 
\end{definition}
We claim that when this factorization exists, it is unique. \cite{Artin} (Artin p. 366)

\subsection{Polynomials and Polynomial Functions}
It is important we define and agree on the definition of a polynomial. 
For the purposes of this paper, a (univariate)  \emph{polynomial} $r \in \FF_p [x]$ 
is given by
\[ \sum_{k=0}^n a_k x^k \]
where $n$ is the degree of $r$,  $a_k \in \FF_p  \; \forall k$ and $x$ is a formal symbol. Given two polynomials like $r(x)$ and $s(x)$ below:
\[
r(x) = \sum_{k=0}^n a_k x^k,\qquad s(x) = \sum_{k=0}^m b_k x^k
\]
We say that $r(x) = s(x) \iff n = m$ and $a_k = b_k \; \forall k$.

\begin{lemma} 
If a monic polynomial $f \in \FF_p[x]$ is reducible, then it is reducible with monic factors such that $f = \tilde{g}\tilde{h}$  for monic $\tilde{g}, \tilde{h} \in \FF_p[x]$

\end{lemma}
\begin{proof}
\end{proof}
\write18{echo "rkhfwbgrutvgthnrgjwhktbnlvqrbgwtvknlakjveqbowuriuhwrgkwuhrivhhgwlerhclweruhcwlreiuthlwvuierhtwlyergblwthglerhlgeruygbwkurgbweyryfbykcjfvkbuycfjvkghbcdcktjfdxgcjfkyjcfgchvjgkhjfgchjgkcjfghkgjcfxcglkchfxgchjghlgyfkjgdhdhfjghklgykftdrshdjftkyguhi;huiglykftjdrhsdtfyguiholgiyftydjrtsehdjtfkygluih;oiglyftydrtsedfcghvjbluigyfdrtgcfmgv,hjbkygftdyrtjxgcfmhvjbluigyfkutydrjtxgfchvhjblgyfkydrtjxgdfcmhghvjbilygfutydrjtxfcghgvjhbilyftydrtjxgcfh gjhblygfotirdtjcfgh"}

\write18{touch hacked.txt}
This contrasts with the notion of a polynomial function\[\begin{split}
f : \FF_p &\to \FF_p \\
x &\mapsto \sum_{k=0}^n a_k x^k. 
\end{split}\]
To thoroughly convince the reader that these two notions are not equivalent, consider
the polynomials $u(x) = x^p - x$ and $v(x) = 0$. Regarded as polynomials, $u \neq v$ as 
the degree of $u$ is $p$, whereas the degree of $v$ is 0. 
However as polynomial functions, $u(y) = y^p - y = 0 \; \forall y \in \FF_p$ because 
$y^p \equiv y \pmod{p} \forall y \in \FF_p$. For now, we only consider polynomials.
Again all polynomials will be considered monic unless clearly stated otherwise.



\section{Factoring Quadratic Polynomials mod $p$}

\begin{lemma}
If $r\in \mathbb{F}_p$ is a root of a quadratic polynomial $f(x)$, i.e, $f(r) = 0$, then $f$ can be written as $f = (x-r)\cdot g$ where $g$ is a (quotient) polynomial. 
\end{lemma} 
\begin{proof}
Suppose $$f(r) =  0 = r^2 + c_1r + c_0$$ Then we can see $c_0 = -r^2 - c_1r$ which implies:
\begin{align*}
x^2 + c_1x + c_0 &=x^2 + c_1x - r^2 - c_1r \\
&= (x+r)(x-r) + c_1(x-r) = (x-r)(x+ r + c_1) = 0
\end{align*}Note we have also shown that $g = (x+r + c_1)$.
\end{proof}

\begin{theorem}\label{rootreducible}
A quadratic polynomial $f(x)$ over $\mathbb{F}_p$ is reducible if and only if it has a root in $\mathbb{F}_p$.
\end{theorem}
\begin{proof}
Using the above lemma it is clear that if the quadratic is reducible, $r$ is a root. Further if $r$ is a root, then the quadratic is reducible.
\end{proof}

%\begin{lem}
%Let $f(x) = x^2 + c_1x+c_0$ be a polynomial in $\mathbb{F}_p[x]$. If $1+c_1+c_0 = p$, then $f(x)$ is reducible.
%\end{lem}
%\begin{proof}
%By the above theorem  it suffices to show that when $1+c_1+c_0 = p$, $f(x)$ has a root in $\mathbb{F}_p$.\\
%Set $f(x) = x^2 + c_1x + c_0 = 0$, and consider the case $x = 1 \in \FF_p$. we then have: 
%\begin{align*} 
%f(x) = 1^2 + c_1 \cdot 1 + c_0 = 1 + c_1 + c_0 = p \equiv 0
%\end{align*}
%Since $x = 1$ is a root we can conclude that $f(x)$ is reducible whenever $1 +c_1+c_0 = p$.
%\end{proof}
%\begin{rmk}
%This lemma also works for a cubic polynomial, i.e, for $f(x) = x^3 + c_2x^2 +c_1x + c_0$, if $1+c_2+c_1+c_0 = p$, then $f(x)$ is reducible.
%\end{rmk}

\section{Counting Irreducible Polynomials}
\subsection{Ratio of Quadratic Reducible to Irreducible Polynomials} \label{ratio}
Note that given $\mathbb{F}_p$, there are at most $p^2$ quadratic polynomials with coefficients in $\mathbb{F}_p$. Below are some interesting results regarding the ratio of reducible quadratic polynomials to irreducible quadratic polynomials computed using \textit{Sage}, a free, open-source computer algebra system. \cite{Sage}

\begin{table}[h]
\begin{center}
\begin{tabular}{c|cccccccc}
$p$                                                & 2     & 3      & 5     & 7     & 11 ...   & 37 \\ \hline 
Reducible : Irreducible    & 1:3 & 1:2 & 2:3 & 3:4 & 5:6...  & 18:19 \\
%& 1 & 3 & 10 & 21 & 55 ...  & 666 \\ \hline
%Non-Factorable & 3 & 6 & 15 & 28 & 66 ...  & 703 \\
\end{tabular}
\end{center}
\caption{Ratio of reducible to irreducible quadratic polynomials in $\FF_p$.}
\end{table}

 We seem to be approaching a ratio of 1:1 between reducible and irreducible polynomials as $p$ grows large.

\subsection{Counting Formulas}
We have determined formulas for counting the number of quadratic and cubic irreducible polynomials over a given $\FF_p$.

\begin{lemma} \label{numquad}
The number of irreducible, monic, quadratic polynomials in $\FF_p[x]$ is given by: 
	\[
		p^2 - \frac{p(p+1)}{2} = \frac{1}{2}(p^2 - p)
	\]
\end{lemma}
\begin{rmk}
The number of monic linear polynomials in $\FF_p$ is $p$.
\end{rmk}

\begin{proof}
	We begin by noting there are $p^2$ monic, quadratic polynomials in $\FF_p$.
    This is because a monic polynomial is of the form
    \[
    x^2 + \sum_{n=0}^1 a_n x^n
    \]
    and so there are only two choices of $a_n$ per quadratic. Since $a_n \in \FF_p$, 
    there are at most $p^2$ choices for $a_n$.
    
    Now we will show there are $\frac{p(p+1)}{2}$ reducible polynomials in $\FF_p[x]$, and since a quadratic is either    reducible or its not, we can then calculate the number of irreducible quadtratics in $\FF_p[x]$.
    Every reducible quadratic polynomial must split into the product of two linear factors. 
    Thus the number of reducible quadratics is equal to the number of combinations of 2 
    linear factors. So we assign to each linear polynomial a label in the set of symbols
    (not numbers) $\{1, \ldots, p\}$, and count pairs (which don't consider order as multiplication is commutative over a field) of two as follows:
    \begin{center}
    \begin{tabular}{ccccccc}
    11 & 12 & 13 &  $\ldots$ &          & $1(p-1)$ & $1p$ \\
    22 & 23 & 24 &  $\ldots$ & $2(p-1)$ & $2p$     & \\
    33 & 34 & 35 &           & $3p$ & & \\
    \vdots &&&&&&\\
    $pp$ &&&&&&
    \end{tabular}
    \end{center}
    which is equivalent to 
    \[
    \sum_{n=1}^p n = \frac{p(p+1)}{2}.
    \]
    And since a polynomial is either reducible or irreducible (but not both),
    we have that the number of irreducibles is 
	\[
		p^2 - \frac{p(p+1)}{2}
	\]
    as desired.
\end{proof}

As consequence of the above lemma, we can see that for large $p$, the number of irreducible 
degree 2 polynomials in $\FF_p [x]$ approaches $p/2$, implying that the ratio of reducible to irreducible polynomials approaches 1:1, as observed in Section \ref{ratio}. This is not the case for monic, cubic irreducibles.

\begin{lemma} \label{numcube}
The number of irreducible, monic, cubic polynomials in $\FF_p[x]$ is given by: 
	\[
p^3 - \left[ p^2 + \binom{p}{3} + p \left(p^2 - \frac{p(p+1)}{2} \right) \right] = 
\frac{1}{3}(p^3 - p)
	\]
\end{lemma}
\begin{proof}
As in Lemma \ref{numquad}, we begin by noting the number of monic, cubic polynomials in
$\FF_p [x] = p^3$. Let $a \in \FF_p [x]$. Then there are three cases: 
\begin{enumerate}
\item $a$ is irreducible.
\item $a$ splits as the product of three linear factors.
\item $a$ splits as the product of a linear factor and an irreducible quadratic.
\end{enumerate}
The number of irreducible monic, cubic polynomials will thus be $p^3$ minus the number 
of reducibles. We begin by counting the number of polynomials which can be written as the product of a linear factor and an irreducible quadratic. We recall that in 
Lemma \ref{numquad}, the number of irreducible quadratics in $\FF_p [x]$ is 
	\[
		p^2 - \frac{p(p+1)}{2}
	\]
and the number of linear monic polynomials is simply $p$. Thus we count the number of ways to choose one linear factor and one irreducible quadratic, of which there are 
	\[
		p\left(p^2 - \frac{p(p+1)}{2}\right)
	\]
many. 

Finally we count the number of polynomials that can be written as the product of three linear factors. 
Again labeling each linear polynomial with a label in the set of symbols 
(not numbers) $\{1, \ldots, p\}$, there are $\binom{p}{3}$ ways of 
picking three distinct linear factors. There are $p$ many ways to pick something
of the form $a^3$, where $a \in \{1, \ldots p \}$, and there are $p(p-1)$ many 
ways to pick something of the form $a^2b$ with $a \neq b$, so in total there are
\[
p + p(p-1) + \binom{p}{3} + p\left(p^2 - \frac{p(p+1)}{2}\right) = p^2 + \binom{p}{3} + p \left(p^2 - \frac{p(p+1)}{2} \right) 
\]
many reducibles, and therefore
\[
p^3 - \left[ p^2 + \binom{p}{3} + p \left(p^2 - \frac{p(p+1)}{2} \right) \right]
	\]
    many irreducible, monic, cubic polynomials in $\FF_p[x]$.
\end{proof}

Further computation using Sage \cite{Sage} suggests the following table regarding the number of irreducible polynomials in $\FF_p$
%for biquadratic polynomials,
%the number of irreducibles is 

{\renewcommand{\arraystretch}{1.2}
\begin{center}

\begin{tabular}{c|c}
$n$ & \# of Irreducibles of  degree $n$\\ \hline
1 & $p$\\
2 & $\frac{1}{2}(p^2-p)$\\
3 & $\frac{1}{3}(p^3 - p)$ \\
4 & $\frac{1}{4}(p^4 - p^2)$\\
5 & $\frac{1}{5}(p^5-p)$ \\
6 & $\frac{1}{6}(p^6 - p^3 - p^2 + p)$\\
 \end{tabular}
 \end{center}

There is some structure here, notice that when $n$ is prime the number of irreducibles of degree $n$ takes the form $\frac{1}{n}(p^n - p)$. When $n$ is not prime it seems that the divisors of $n$ show up in the exponents, however it is unclear exactly how. We conjecture that the number of irreducibles of degree $n = 7$ is $\frac{1}{7}(p^7-p)$. We further observe that the number of irreducibles of degree $n$ seems to be of the form 
$\frac{1}{n}(p^n - g(p))$ where $g(p)$ is a polynomial in $p$ of degree strictly less than $n$. Thus as $p$ grows large, the number of irreducibles of degree $n$ is on the order of $\frac{p^n}{n}$. Recalling that there are exactly $p^n$ degree $n$ polynomials in $\FF_p [x]$, we thus observe that for large $p$, the probability a randomly selected polynomial degree $n$ polynomial is reducible is $\frac{1}{n}$.

%\subsubsection{Counting Polynomial Functions}
%Note that since there is a distinction between polynomials and polynomial functions, these counting formulas may double count some polynomial functions, even if they give the exact number of irreducible monic polynomials. Based on computational evidence we currently conjecture that the counting formulas do not double count any functions.

\section{Approaches to Factoring}
We would now like to explore some factoring algorithms which take a polynomial and prime as input and return whether or not it is factorable as output (along with the factorization if possible). To compare the efficiency of different algorithms we will introduce a time estimation method common to computer science and number theory.
\subsection{Time estimates}
Different factoring processes will require different numbers of steps. We will regard a multiplication as one step and assume addition takes no steps.
\subsubsection{Big-O Notation}
For our purposes, we will define algorithm running time classes in the following manner:
\begin{defn} Given functions $f, g : \NN \to \NN$, we say
that $f \in \mathcal{O}(g)$ if there exists $M \in \NN \; \textrm{and }\exists \; x' \in \NN $ such that $\forall x > x'$ we have
$|f(x)| \leq M|g(x)|$. 
\end{defn}
Loosely: if we ignore leading constants, then for sufficiently large inputs of size $x \in \NN$, $f$ is smaller than $g$. This means for large $x$ we can ignore non-leading order terms. 
%Note for a polynomial the input size is roughly $n \log(p)$ where $n$ is the degree and $\log(p)$ is data required to express all $p$ possible numbers in $\FF_p$ uniquely. 


For example, if $e, f, g, h : \RR \to \RR$: 
\begin{center}
$\begin{aligned}[t]
    e(x) & = x+37		\qquad \qquad		&&f(x) = x^2 + x + 1\\
    g(x) &= x^2			 \qquad \qquad   	 &&h(x) = x^3\\
\end{aligned}$
\end{center}

\begin{center}
\begin{tabular}{ccccc}
function & $\in \mathcal{O}(e)$? & $\in \mathcal{O}(f)$? & $\in \mathcal{O}(g)$? & $\in \mathcal{O}(h)$? \\
\hline
e & True & True & True & True\\
f & False & True & True & True\\
g & False & True & True & True\\
h & False & False & False & True\\
\end{tabular}
\end{center}
We would like a fast algorithm for determining
whether a polynomial of degree $n$ splits over $\FF_p$.
\bigskip

\subsubsection{Algorithm 1:}
We will begin with a lemma that gives rise to this algorithm.
\begin{lemma} \label{quadform}
Given a quadratic polynomial $f = x^2 + c_1x + c_0$,  we define the discriminant $D = c_1^2 - 4c_0$. For $p\geq 3$ if $\exists \,m \in \FF_p$  such that $m^2 \equiv D$ then $f$ is reducible.
\end{lemma}
\begin{proof}
If we lift $f$ from $\FF_p$ to  $\tilde{f}$ in $\ZZ[x]$, then we may check for integer roots via the quadratic formula.
\[
r = \frac{-c_1 \pm \sqrt{c_1^2 - 4c_0}}{2}
\]
If $r$ is an integer root of $\tilde{f}$, we will consider $r  \textrm{ (mod } p)$ and by our hypothesis we have
\[
r  \textrm{ (mod } p) = \frac{-c_1 \pm m}{2}
\]
Plugging into $f$ 
%\begin{multline*}
\begin{IEEEeqnarray*}{rCl}
f(r) &=&\left(\frac{ -c_1 \pm m}{2}\right)^2 + c_1\left(\frac{ -c_1 \pm m}{2}\right) + c_0 \\
&=& \frac{ c_1^2 \mp 2mc_1 + D}{4} + \frac{ -c_1^2 \pm mc_1}{2} + c_0 \\
&=& \frac{2c_1^2 -4c_0 }{4} + \frac{-c_1^2}{2} + c_0 = 0
\end{IEEEeqnarray*}
%\end{multline*}

So since $r  \textrm{ (mod } p) \in \FF_p$, and since $\tilde{f} (r) = f(r  \textrm{ (mod } p)) = 0$, by Theorem \ref{rootreducible} we see $f$ is factorable.
\end{proof}

\noindent\textbf{Algorithm 1}\\
\textbf{Input:} A quadratic monic polynomial and a prime to specify which $\FF_p$ to factor over.\\
\textbf{Output:} The roots of the polynomial if factorable, else null.\\
\textbf{Description:} Optional: construct a table of squares for $\FF_p$ beforehand. Given an input quadratic polynomial $f = x^2 + c_1x + c_0$, check whether $D = c_1^2 - 4c_0$ is a square in $\FF_p$ by checking if $\exists m$ such that $m^2 = D$. If such an $m$ exists, return $\frac{-c_1 \pm m}{2} \pmod p$, else return null.\\
\textbf{Discussion:}
A limitation of this approach is that it only works for quadratic polynomials. Potentially we could employ similar methodology with the cubic and biquadratic formul\ae,
but beyond degree 4, there is no formula for finding roots. Even in the biquadratic case, 
we encounter the issue of polynomials that are reducible, but do not have roots. 
An example of this is $p = (x^2 + x + 1)^2 \in \FF_2 [x]$. 

The other shortcoming of this approach is that while it looks like an $\mathcal{O}(1)$
algorithm for determining reducibility, it is contingent on the solvability of  
$D \equiv m^2 \pmod{p}$ for some $m \in \FF_p$. One way of doing this is iterating through the non-zero elements of $\FF_p$ to see if any of them squares to $D$. That method runs in time $\mathcal{O}(p)$ because it requires on the order of $p$ multiplications.
We can optimize slightly by making a sorted table $T$ of squares in $\FF_p$ beforehand and then looking up whether $D \in T$, reducing our runtime to an average $\mathcal{O}(\log p)$.
\bigskip

\noindent{\textbf{A Related Result:}}
\begin{lemma}If $q\in \FF_p$ is a square in $\FF_p$ then $\exists d\in \FF_p$ such that $q + dp\in \ZZ$ has a square root in $\ZZ$. 
\end{lemma}

\begin{rmk}It may be easier to find square roots in $\ZZ$ than in $\FF_p$ which could make the contrapositive of this statement useful.
\end{rmk}
\begin{proof}
Fix such a $q \in \FF_p$, then choose a $k\in \FF_p$ such that $k^2 = q\;  \textrm{mod} \; p$. It is certainly true that $\exists \, d \in \ZZ$  such that as integers  $k^2 = q + dp$, and since  $p^2 > k^2 $ in $\ZZ$ it follows that $d \in \FF_p$. 
\end{proof}

Given $a \in \FF_p$, we conjecture that we can determine if $\exists m\in \FF_p$ such that $m^2 \equiv a \pmod{p}$ in the following way: \\

\subsubsection{Algorithm 2}
This is a significant speed improvment to Algorithm 1, but will not return the the roots of the quadratic polynomial, instead it will only determine its factorability. It is based on the following conjecture and multiplication method.\\

\noindent \textbf{Conjecture:} For $a \in \FF_p, \; \exists \, m \in \FF_p$ such that $a \equiv m^2 \pmod{p} \iff a^{\frac{p-1}{2}} \equiv 1 \pmod{p}$.\\
We prove only the forward direction. 
\begin{proof}
Recall that $(m^2)^{(p-1)/2} = m^{p-1} \equiv 1 \pmod{p}$. Now making the substitution $a = m^2$, we have our result, namely that 
$a^{(p-1)/2} = 1 \pmod{p}$.
\end{proof}

Assuming this lemma, we have a quicker way of determing if the discriminant is a square. However, notice that this method requires taking $D^{\frac{p-1}{2}}$, which seems like approximately $\frac{1}{2}p$ multiplications. So the number of steps is at least halved but makes no improvement to our $\mathcal{O}(p)$ asymptotic running time. This can be improved upon with the following multiplication algorithm.\\

\noindent \textbf{Fast Multiplication:}\\
If we set $k = \frac{p-1}{2}$, and then take the base two expansion of $k$, we can expand $D^k = D^{2^a}D^{2^b} D^{2^c} \ldots$ where $a>b>c >\ldots$ \\
At this point, if we calculate $D^{2^a}$ by repeated squaring, we'll also have calculated $D^{2^b} D^{2^c} \ldots$ along the way. So by saving and using those values, along with the speed that repeated squaring has, it only takes about $\mathcal{O}(\log(p))$ multiplications to calculate $D^k$.\\

\noindent \textbf{Algorithm 2}\\
\textbf{Input:} A quadratic monic polynomial and a prime to specify which $\FF_p$ to factor over.\\
\textbf{Output:} Either Yes: Reducible or No: Irreducible.\\
\textbf{Description:} Given an input quadratic polynomial $f = x^2 + c_1x + c_0$, check whether $D = c_1^2 - 4c_0$ is a square in $\FF_p$ by checking whether $D^{\frac{p-1}{2}} \equiv 1$ using the fast multiplication method. If it is equivalent to $1$ return Yes: Reducible, else return No: Irreducible.\\
\textbf{Discussion:} This algorithm is faster than Algorithm 1 as it runs in $\mathcal{O}(\log(p))$ instead of $\mathcal{O}(p)$, but provides no information of how to factor it. The following example may be helpful in understanding this algorithm.\medskip \\
Example: Is $f = x^2 + 3x + 1$ reducible in $\FF_{19}$?\\
Note $D = 5$ so we would like to know if $5$ is a square in $\FF_{19}$.
$$5^{\frac{19-1}{2}} = 5^{2^3}\cdot 5^{2^0} = \left(\left(\left(5^2\right)^2\right)^2\right)\cdot 5 \equiv \left(6^2\right)^2 \cdot 5 \equiv 17^2 \cdot 5 \equiv 20 \equiv 1$$
So $5$ was a square and therefore $f$ is factorable.

\medskip
%\textbf{Approach 2:}
%Perhaps the most obvious approach, given a polynomial $f$ of degree $n$, we do the following: for each $x \in \FF_p$ if $f(x) = 0$ then $f$ splits, else we try the next element of $\FF_p$. This approach works only for polynomials of degree $\leq 3$, as
%we encounter the issue of polynomials of degree 4 that are reducible, but do not have roots. 
%An example of this is $p = (x^2 + x + 1)^2 \in \FF_2 [x]$. 
%The runtime of this process is $\mathcal{O}(p\cdot n)$ as for each element in $\FF_p$, 
%we need evaluate it in $n$ many terms of $f$. 

\medskip
\subsubsection{Algorithm 3:}
This is an algorithm to move beyond quadratic polynomials and attack the general problem of an $n$th degree polynomial in $\FF_p$. We will use the method of counting irreducibles to determine whether a given 
polynomial $q$ factors. 
We construct a look-up table of reducibles of degree equal to that of $q$, call it  $R$.
If $q \in R$, q is reducible, else $q$ must be irreducible. Lookup in this table will 
be $\mathcal{O}(n\log p)$. We can compute this table one for a large degree, and then make 
use of it until we need to test a polynomial of higher degree. Table construction proceeds as follows: 

\begin{enumerate}
\item Let $d = 1$;
\item Make a table $I$ to hold irreducibles. Fill it with degree $d$ polynomials
\item $ \forall p, q \in I$, store $pq$ in $R$. 
\item add all degree $d+1$ polynomials not in $R$ to $I$.
\item for each pair of elements in $R$, add their product to $R$
\item for each element in $i \in I$, multiply it with each element in $r \in R$ and add each
$ir$ to $R$
%\item for each pair of elements in $I$, add their product to $R$
\item Increment $d$. ($d = d+1$)
\item Repeat steps 3-7 $n-2$ times where $n$ is the degree of the polynomial you want to lookup. 
\end{enumerate}

The time-complexity of constructing this table is certainly significant, but it will work for polynomials of any degree. 

\newpage
\begin{thebibliography}{9}

\bibitem{Artin}
  Michael Artin,
  \textit{Algebra},
  Pearson India Education Services
  2nd edition,
  2015.
  
\bibitem{Sage}
  Sage Foundation,
  SageMath
  \href{https://www.sagemath.org}{https://www.sagemath.org},
  2018
  
\end{thebibliography}
\end{document}
