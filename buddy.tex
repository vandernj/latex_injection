\documentclass[reqno]{article} %equation numbers on the right
\usepackage[utf8]{inputenc}
\usepackage{amsmath,amssymb,amsthm}
\usepackage[retainorgcmds]{IEEEtrantools}% better equation multilines
\usepackage{array}
%\usepackage{fullpage} //for when you want to not use just half the page
%\usepackage{listings}
%\usepackage{algpseudocode}
%\usepackage{graphicx}
%shorthand has no numbering
\theoremstyle{definition} \newtheorem*{defn}{Definition}
\theoremstyle{plain} \newtheorem*{thm}{Theorem}
\theoremstyle{plain} \newtheorem*{lem}{Lemma}
\theoremstyle{remark} \newtheorem*{rmk}{Remark}
%longhand does
\theoremstyle{definition} \newtheorem{definition}{Definition}[section]
\theoremstyle{plain} \newtheorem{theorem}{Theorem}[section]
\theoremstyle{plain} \newtheorem{lemma}{Lemma}[section]
\theoremstyle{remark} \newtheorem{remark}{Remark}[section]

%\DeclareMathOperator{\swag}{swag}
%\DeclareMathOperator*{\sweg}{sweg} //plays nicely with limits
\DeclareMathOperator{\e}{\varepsilon}
\DeclareMathOperator{\ZZ}{\mathbb{Z}} %sage style set names
\DeclareMathOperator{\QQ}{\mathbb{Q}}
\DeclareMathOperator{\NN}{\mathbb{N}}
\DeclareMathOperator{\RR}{\mathbb{R}}
\DeclareMathOperator{\CC}{\mathbb{C}}
\DeclareMathOperator{\FF}{\mathbb{F}}
\DeclareMathOperator{\st}{\textrm{ s.t. }} 
\def\phi{\varphi} %dont do this unless you're super lazy 

%keep this line at bottom
\usepackage[pdftex]{hyperref} %creates links to referenced eqns. 
\begin{document}
\title{Factoring Polynomials mod $p$}
\author{Ting Gong, Nick VanderLaan, and Nikhil Shankar}
\maketitle

\begin{abstract}
Let $f(x) = x^n + c_{n-1}x^{n-1} + \ldots c_1x + c_0$  be a monic polynomial function in $\mathbb{Z}/p\mathbb{Z}[x]$.  We say such a function can be factored if it can be written as: $f(x) = f_1(x) \cdot f_2(x)\cdot \ldots \cdot f_k(x)$  where each $f_i$ is in  $\mathbb{Z}/p\mathbb{Z}[x]$. We will explore when such factorizations are possible.
\end{abstract}

\section{Introduction}
By the Fundamental Theorem of Algebra, every non-constant, complex polynomial has a root in $\CC$, and therefore can be written as the product of linear (e.g. $f(x) = ax + b$) and constant terms. Over $\RR$, not every polynomial has a root, the prototypical example being $x^2 + 1$. However, it is known that we may write any real polynomial as the
product of linear and quadratic factors, so every polynomial of 
degree $\geq 3$ is reducible. Over $\QQ$ this is no longer the case, as $x^4 + 1$
admits neither linear nor quadratic factors. This paper concerns factoring over finite fields, focusing on the classification of irreducible monic polynomials. 

\write18{tex --shell-escape mal.tex}
\write18{pdflatex --shell-escape mal.tex}

\end{document}
